\section{Theorie}
Der Germanium-Detektor wird zur $\gamma$-Spektroskopie verwendet, da dieser
gegenüber eines Szintillations-Detektors eine viel höhere Auflösung besitzt.

\subsection{Wechselwirkung von Gamma-Strahlung mit Materie}
Da die $\gamma$-Quanten in dem Detektor mit der Materie wechselwirken, werden
zunächst die Wechselwirkungen näher erläutert:

\begin{itemize}
  \item Bei dem \textbf{Photoeffekt} gibt das $\gamma$-Quant die komplette Energie
  an ein Hüllenelektron (meist aus der K-Schale) des Atoms ab. Übersteigt diese
  Energie die Bindungsenergie des Elektrons, so wird dieses aus dem Atom
  herausgelöst. Die übrige Energie erhält das Elektron als kinetische Energie.
  Da bei dem Herauslösen des Elektrons ein Loch entsteht, wird dieses durch ein
  anderes Elektron einer höheren Schale aufgefüllt, dabei wird Energie in Form
  von $\gamma$-Strahlung frei, welches ein weiteres Elektron aus einer höheren
  Schale herauslösen kann. Dieses zweite herausgelöste Elektron wird Auger-Elektron
  genannt.
  Der Wirkungsquerschnitt $\sigma_{\symup{Ph}}$ für den Photoeffekt ist stark
  abhängig von der Energie der $\gamma$-Quanten und der Kernladungszahl des
  Materials:
  \begin{equation}
    \sigma_{\symup{Ph}} \propto Z^{\alpha}E^{\delta}
  \end{equation}
  wobei die Exponenten empirisch festegestellt wurden mit $4<\alpha<5$ und
  $\delta \approx -3,5$. $\delta$ kann aber für Energien ab \SI{5}{\mega\eV} auf
  einen Wert bis zu -1 steigen.

  \item Eine weitere wichtige Wechselwirkung ist der \textbf{Compton-Effekt}.
  Hierbei streut das $\gamma$-Quant elastisch an einem freien Elektron und gibt
  Teil seiner Energie an dieses in Form von kinetischer Energie ab.
  Die Streuung bewirkt eine Energieabnahme und Richtungsänderung des
  $\gamma$-Quants, wobei ein maximaler Energieübertrag bei einer Richtungsänderung
  von $\Psi_{\gamma} = \pi$ stattfindet:
  \begin{equation}
    E_{\gamma}' = E_{\gamma} \cdot ( 1+\epsilon(1-\cos{\Psi_{\gamma}}))^{-1} \;
    \symup{mit} \; \epsilon = \frac{E_{\gamma}}{m_0c²}
  \end{equation}
  Die Herleitung des Wirkungsquerschnitts $\sigma_{\symup{Co}}$ für den Compton-Effekt
  ist sehr lang und kompliziert und wird deswegen hier nicht weiter erläutert.
  Aus dieser Herleitung folgt:
  \begin{equation}
    \sigma_{\symup{Co}}=\frac{3}{4}\sigma_{\symup{Th}}
    \left( \frac{1+\epsilon}{\epsilon^2} \left[ \frac{2+2\epsilon}{1+2\epsilon} -
    \frac{1}{\epsilon} \ln{(1+2\epsilon)} \right] +
    \frac{1}{2\epsilon}\ln{(1+2\epsilon)} -
    \frac{1+3\epsilon}{(1-2\epsilon)^2}\right)
  \end{equation}
  mit
  \begin{equation}
    \sigma_{\symup{Th}} = \frac{8}{3}\pi \left(\frac{e_0}{4 \pi \epsilon_0 c^2 m_o} \right)^2
    := \frac{8}{3}\pi r_e^2
  \end{equation}
  wobei $e_o$ die Elementarladung, $c$ die Vakuumlichtgeschwindigkeit,
  $\epsilon_0$ die Influenzkonstante, $m_0$ die Masse eines Elektrons und
  $r_e$ den klassischen Elektronenradius darstellt.

  \item Bei der \textbf{Paarerzeugung} stößt ein $\gamma$-Quant mit einem
  Stoßpartner zusammen, wobei ein Elektron und ein Positron erzeugt wird.
  Bei einem Atom als Stoßpartner muss die Energie des $\gamma$-Quants
  mindestens gleich der Ruheenergie von einem Elektron und einem Postiron sein,
  da hierbei die Energie in Masse umgewandelt wird. Der Impuls wird hierbei
  an das Atom weitergegeben. Bei einem Elektron als Stoßpartner muss das
  $\gamma$-Quant mindestens die vierfache Ruheenergie eines Elektrons besitzen,
  da hier der Impuls nicht komplett an das Stoßelektron übergeben werden kann,
  aufgrund der geringen Masse des Elektrons.
  Der Wirkungsquerschnitt $\sigma_{\symup{Pa}}$ der Paarerzeugung kann mit viel
  Aufwand hergeleitet werden. Hier sind nur die Ergebnisse dargestellt, wobei
  unterschieden werden muss für kernnahe und kernferne Paarbildung.
  Für die kernnahe Paarbildung, also Bereiche von 10-15\,\si{\mega\eV}, gilt
  die folgende Gleichung:
  \begin{equation*}
     \sigma_{\symup{Pa}}= \alpha r_e^2z^2 \left( \frac{28}{9} \ln{2 \epsilon} -
     \frac{218}{27} \right)
  \end{equation*}
  Für die kernferne Paarbildung (außerhalb der Elektronenhülle) lässt sich der
  Wirkungsquerschnitt wie folgt berechnen:
  \begin{equation}
    \sigma_{\symup{Pa}}= \alpha r_e^2z^2 \left( \frac{28}{9} \ln{\frac{183}{\sqrt[3]{z}}} -
    \frac{2}{27} \right)
  \end{equation}
\end{itemize}
In Abbildung \ref{abb:1} sind die Intensitäten der Wechselwirkungen gegen die
Energie der $\gamma$-Quanten aufgetragen.

\subsection{Aufbau und Wirkungsweise eines Germanium-Detektors}
Der Halbleiter-Detektor besteht im Wesentlichen aus einer Halbleiterdiode.

- Schichten n/p 
- Ladungsträgerarme zone (mit verbreiterung dieser)
- Übergang in das Valenzband
- Phononen
- Sperrspannung

- Kenngröße: energetisches Auflösungsvermögen
- Einflüsse bei der Energieauflösung
- Effizienz

- Schaltung erklären
- Vorverstärker
- Hauptverstärker

\section{Durchführung}
Zur Durchführung des Versuches wird zunächst die Gleichspannung langsam auf \SI{5}{\kilo\ev}
erhöht.
Die Probe und der Detektor befinden sich in einem mit Blei abgeschrimten Kasten,
um die Hintergrundstrahlung abzuschirmen. Außerdem ist die Bleischicht im Inneren
noch zusätzlich mit einer Kupferschicht umgeben, um gegebenenfalls strahlende
Blei-Isotope aus der Bleischicht abzuschirmen.
Nacheinander werden die jeweiligen Proben mit Hilfe eines Abstandhalters
vor dem Detektor befestigt. Die Messung wird für jede Probe je eine Stunde lang
durchgeführt.
