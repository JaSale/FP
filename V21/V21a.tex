\section{Zielsetzung}
Im Versuch sollen durch das Verfahren des optischen Pumpens die Energieaufspaltungen
in der Hyperfeinstruktur oder der Zeeman-Aufspaltung mit hoher Präzision gemessen werden.
Konkret werden die Energieaufspaltungen zweier Rubidium-Isotope ermittelt.

\section{Theorie}
Während die Besetzung der innersten Elektronenschalen nach dem Pauliprinzip vollständig besetzt
wird, so erfolgt die Besetzung der äußeren Schalen temperaturabhängig.
Für die zwei Zustände mit den Energien $W_1$ und $W_2$ und den Besetzungszahlen $N_1$ und
$N_2$ gilt demnach folgendes Verhältnis zwischen den Besetzungszahlen:
\begin{align*}
    \frac{N_{2}}{N_{1}} = \frac{g_{2}}{g_{1}}\frac{\exp(-W_{2}/\text{kT})}{\exp(-W_{1}/\text{kT})} \; .
\end{align*}
Die Größen $g_{\text{i}}$ geben hierbei die Anzahl der zu den Energien $W_1$ und $W_2$ gehörenden
Zustände an.
Mittels des Verfahrens des optischen Pumpens ist es möglich, dieses Verhältnis zu verändern
oder gar zu invertieren ($N_{2} > N_{1}$).
Zunächst soll im Folgenden betrachtet werden, wie sich die Energieniveaus eines Atoms aufspalten können.

\subsection{Energieaufspaltung}
In einer ersten Näherung soll zunächst der Fall betrachtet werden, in dem davon ausgegangen wird, der
Kernspin verschwindend ist.
Der Gesamtdrehimpuls der Valenzelektronen des Atoms $\vec{\text{J}}$ erzeugt hierbei
ein magnetisches Moment $\vec{\mu}_{\text{J}}$, welche in folgendem Zusammenhang stehen:
\begin{align*}
    \vec{\mu}_{\text{J}} &= -\text{g}_{\text{J}} \cdot \mu_{\text{B}} \cdot \vec{\text{J}} \\
    |\vec{\mu}_{\text{J}}| &= \hspace{1em} \text{g}_{\text{J}} \cdot \mu_{\text{B}} \cdot \sqrt{ \text{J} (\text{J}+1) }
\end{align*}
Hier beschreibt $\mu_{\text{B}}$ das Bohr'sche Magneton.
Der Landé-Faktor $\text{g}_{\text{J}}$ dient zur Berücksichtigung des Umstandes, dass sich der Gesamtdrehimpuls aus dem Bahndrehimpuls $\vec{\text{L}}$ und dem Spin $\vec{\text{S}}$ des Valenzelektrons zusammensetzt.
Für die magnetischen Momente, die durch Spin und Bahndrehimpuls hervorgerufen werden $\vec{\mu}_{\text{L,S}}$, gilt:
\begin{align*}
    |\vec{\mu}_{\text{L}}| &= \mu_{\text{B}} \sqrt{\text{L} (\text{L}+1)} \\
    |\vec{\mu}_{\text{S}}| &= \text{g}_{\text{S}} \cdot \mu_{\text{B}} \sqrt{\text{S}(\text{S}+1)} \; .
\end{align*}
Vektoriell beschrieben, sieht die Kopplung der magnetischen Momente, die aus dem Spin und dem Bahndrehimpuls resultieren wie folgt aus:
\begin{align*}
    \vec{\mu}_{\text{J}} &= \vec{\mu}_{\text{L}} + \vec{\mu}_{\text{S}} \\
    |\vec{\mu}_{\text{J}}| &= |\vec{\mu}_{\text{L}}| \cos(\beta) + |\vec{\mu}_{\text{S}}| \cos{\alpha} \; .
\end{align*}
Nun lässt sich mit Hilfe des Cosinussatzes, angewendet auf das Dreieck, das von $\vec{\mu}_{\text{L}}, \vec{\mu}_{\text{J}} \text{und} \vec{\mu}_{\text{S}}$ aufgesapnnt wird, eine Formel zur Berechnung des Landé-Faktors herleiten:
\begin{align}
    \text{g}_{\text{J}} = \frac{(\text{g}_{\text{s}} + 1) \cdot \text{J} (\text{J} + 1) + (\text{g}_{\text{s}}-1) \cdot \left[\text{S}
    \cdot (\text{S}+1) - \text{L} (\text{L} + 1) \right]}{2 \cdot \text{J} (\text{J} +1 )}
    \label{eq:La-Fa.gj}
\end{align}
Durch Anlage eines äußeren Magnetfeldes B, kommt es nun zur Aufspaltung der Energieniveaus, was als Zeeman-Effekt bekannt ist. Die Energie der magnetischen Wechselwirkung wird wie folgt beschrieben:
\begin{align}
    E_{\text{mag}} &= \vec{\mu}_{\text{J}} \cdot \vec{\text{B}} \\
    &= \text{M}_\text{J} \cdot \text{g}_\text{J} \cdot \mu_{\text{B}} \cdot \text{B} \; .
    \label{eq:E_mag}
\end{align}
Hierbei gilt für die Orientierungsquantenzahl $\text{M}_\text{J}$ auf Grund der Richtungsquantelung: $\text{M}_{\text{J}} \in [-\text{J}, -\text{J}+1,\hdots,\text{J}-1,\text{J}]$ .

\noindent Nun sollen die fälle betrachtet werden, in denen der Kernspin nicht verschwindet.
Bei hinreichend schwachem Magnetfeld, wovon im Falle des Versuches ausgegangen werden kann, koppeln der Gesamtdrehimpuls der Elektronenhülle $\vec{\text{J}}$ und der Kernspin $\vec{text{I}}$ vektoriell aneinander zu einem Gesamtdrehimpuls des Systems $\vec{\text{F}}$:
\begin{align*}
    \vec{\text{F}} = \vec{\text{I}} + \vec{\text{J}} \; .
\end{align*}
Durch den nicht mehr verschwindenden Kernspin, spalten sich die Energieniveaus nun in die Hyperfeinstruktur auf.
Die Anzahl der Niveaus wird durch $2\text{J}+1$ oder $2\text{I}+1$ ermittelt, je nachdem, ob J größer als I ist oder anders herum.
Die Quantenzahl F beschreibt dabei, wie sich die einzelnen Niveaus aufspalten und läuft von I+J bis $|\text{I}-\text{J}|$.
Die Hyperfeinstruktur kann wiederum in $2\text{F}+1$ Zeeman-Niveaus aufgespalten werden, indem ein äußeres Magnetfeld angelegt wird.
Eine Beispielhafte Aufspaltung ist in Abbild \ref{abb:Aufspaltung} zu sehen.
\FloatBarrier
\begin{figure}
    \centering
    \includegraphics[width=0.7\textwidth]{Aufspaltung.PNG}
    \caption{Aufspaltung der Energieniveaus eines Alkaliatoms mit I = $\frac{3}{2}$ und J = $\frac{1}{2}$ auf Grund von Elektronenspin, Kernspin und externem Magnetfeld. \cite{Q1}}
    \label{abb:Aufspaltung}
\end{figure}
\FloatBarrier
Zwischen zwei benachbarten Energieniveaus herrscht eine Energiedifferenz, die wie folgt berechnet werden kann:
\begin{align*}
    E_{\text{HF}} = \text{g}_{\text{F}} \mu_{\text{B}} \text{B} \; .
\end{align*}
Mittels vektorieller Betrachtung lässt sich der Landé-Faktor für die Zeeman-Aufspaltung bestimmen:
\begin{align*}
    |\vec{\mu}_{\text{F}}| &= \text{g}_{\text{F}} \cdot \mu_{\text{B}} \sqrt{\text{F} (\text{F}+1)} \\
    &=\text{g}_{\text{J}} \cdot \mu_{\text{B}} \sqrt{\text{J} (\text{J} + 1)} \cdot \cos\left(\measuredangle(\vec{\text{J}},\vec{\text{F}})\right) +
    \text{g}_{\text{I}} \cdot \mu_{\text{K}} \sqrt{ \text{I}(\text{I}+1) } \cdot \cos\left(\measuredangle(\vec{\text{I}},\vec{\text{F}})\right) \; .
\end{align*}
Hierbei beschreibt $\mu_{\text{K}}$ das magnetische Moment des Kerns und $\text{g}_{\text{I}}$ den zugehörigen Landé-Faktor.
Allerdings ist auf Grund der großen Massendifferenz zwischen Elektron und Nukleon ( $\rightarrow \mu_{\text{K}} \ll \mu_{\text{B}} $ ) zu beachten, dass in obiger Gleichung der zweite Summand zumeist vernachlässigt wird, womit für den Landé-Faktor folgendes gilt:
\begin{align}
    \text{g}_{\text{F}} \approx \text{g}_{\text{J}} \frac{\text{F} (\text{F} + 1)+\text{J} (\text{J} + 1) - \text{I}(\text{I} + 1)}{2\text{F}(\text{F}+1)} \; .
	\label{eq:La-Fa.gf}
\end{align}

\subsection{Optisches Pumpen}
Wie bereits erwähnt, kann die Besetzung höherer Energieniveaus mit Hilfe der Boltzmannstatistik beschrieben werden.
Mit Hilfe des Verfahrens des optischen Pumpens ist es möglich diese Verteilung zu invertieren, also eine größere Bestzung eines höheren Energiezustandes zu erzeugen.
Um von einem Atom absorbiert zu werden, benötigt ein Photon folgende Energie:
\begin{align*}
    hf = W_{2} - W_{1}
\end{align*}
Diese Energie entspricht der Energie des Photons, was bei einem Übergang von $W_{2}$ auf $W_{1}$ frei würde.
Zunächst soll der Einfachheit halber ein Alkaliatom betrachtet werden, da dies lediglich ein Valenzelektron und keinen Kernspin besitzt.
Der Grundzustand dieses Atoms ist dann ${}^2S_{{}^1\!/\!_2}$ und die ersten beiden angeregten Zustände sind dann ${}^2P_{{}^1\!/\!_2}$ und ${}^2S_{{}^3\!/\!_2}$. Wie in Abbildung \ref{abb:Aufspaltung2} zu sehen, sind zwei Übergänge möglich: $D_{1}$ und $ D_{2} $.
Der Gesamtspin J der Zustände ${}^2S_{{}^1\!/\!_2}$ und ${}^2P_{{}^1\!/\!_2}$ beträgt $\text{J}= {}^1\!/\!_2$. Daraus ergeben sich für die Orientierungsquantenzahl $\text{M}_\text{J}$ folgende mögliche Werte für die Aufspaltung und für die Differenz $\Delta \text{M}_\text{J}$:
\begin{align*}
    \text{M}_\text{J} &= \pm \frac{1}{2} \\
    \Delta \text{M}_\text{J} &= 0, \pm 1 \; .
\end{align*}
\FloatBarrier
\begin{figure}
    \centering
    \includegraphics[width= 0.49\textwidth, height=0.3\textwidth]{Aufspaltung2.PNG}
    \includegraphics[width=0.40\textwidth, height=0.3\textwidth]{Aufspaltung3.PNG}
    \caption{Dublettstruktur des Alkali-Atoms mit zugehörigen Quantenzahlen, sowie die Zeeman-Aufspaltung des Grundniveaus und des ersten angeregten Zustands mit möglichen Übergängen. \cite{Q1}}
    \label{abb:Aufspaltung2}
\end{figure}
Je nach dem, welcher Übergang stattfindet, ist das dabei emittierte Licht unterschiedlich polarisiert.
Beim $\sigma^+$-Übergang ( $\Delta \text{M} = +1$ ) ist der Spin der emittierten Lichtquanten antiparallel zur Ausbreitungsrichtung und es entsteht rechts zirkular-polarisiertes Licht. Beim $\sigma^-$-Übergang ( $\Delta \text{M} = -1$ ) hingegen sind Spin und Ausbreitungsrichtung parallel. Die beiden $\sigma$-Übergänge werden entlang des Magnetfeldes emittiert, wohingegen der $\pi$-Übergang ( $\Delta \text{M} = +0$ ) senkrecht zum Magnetfeld abgestrahlt wird, wo das Intensitätsmaximum auftreten, und linear polarisiertes Licht emittiert wird.
Nun besteht die Möglichkeit ein Gas aus diesem hypotetischen Alkali-Atom, das sich im thermischen Gleichgewicht befindet,
mit zirkular polarisiertem Licht ($D_1$-Licht) anzuregen und ein äußeres Magnetfeld anzulegen. Da die Orientierungsquantenzahl M hierbei nur +1 oder -1 betragen kann, sind nur Übergänge von ${}^2S_{^1\!/\!_2}$ mit  $M_J=-^1\!/\!_2$ nach ${}^2P_{^1\!/\!_2}$, $M_J=+^1\!/\!_2$ möglich.
Der Übergang vom ersten angeregten Zustand ${}^2P_{^1\!/\!_2}$ mit $M_J=+^1\!/\!_2$ zurück in den Grundzustand findet hierbei durch spontane
Emission statt. Dabei wird sowohl der Grundzustand mit $M_J=-^1\!/\!_2$
als auch der mit $M_J=+^1\!/\!_2$ besetzt. Diese beiden Vorgänge sorgen dafür, dass der energetisch niedrigere Zustand entgegen der thermischen Verteilung ohne Anregung und Magnetfeld, immer weniger besetzt ist und sozusagen "leer" gepumpt wird.
Je weniger der energetisch niedrigere Zustand besetzt ist, umso weniger Lichtquanten werden hier dann absorbiert und das anregende $D_1$-Licht
kann vollständig mit einem Photoddetektor gemessen werden. Die gemessene
Transmission nähert sich asymptotisch dem Wert 1, der nur erreicht werden kann, falls es gelingt, den niedrigeren Zustand mit $M_J=-^1\!/\!_2$ völlig leer zu pumpen.

\subsection{Resonanzstellen}
In Abhängigkeit vom angelegten Magnetfeld existieren für das Alkali Atom zwei Resonanzstellen, die in Abbildung \ref{abb:Resonanz} zu sehen sind. Die erste liegt bei $\text{B}=0$, was daran liegt, dass es durch das Fehlen des Magnetfeldes zu keiner Aufspaltung der Spektrallinien und somit auch zu keinen induzierten Übergängen zwischen den Niveaus kommt und folglich auch das optische Pumpen nicht funktionieren kann.
In diesem Bereich ist der Einfluss des Erdmagnetfeldes zu bestimmen und mit dem angelegten Magnetfeld auszugleichen.
Wird das Magnetfeld eingeschaltet, so findet das optische Pumpen statt und es kommt zur oben beschriebenen Besetzungsinversion.
Im Falle, dass die Energie eines eintreffenden Photons der Energiedifferenz, die es zur Zeeman-Aufspaltung braucht, entspricht, so kommt es hier wieder zur induzierten Emission, was wiederum dazu führt, dass das energieärme Besetzungsniveau ${}^2S_{{}^1\!/\!_2}$, $M=-^1\!/\!_2$ wieder aufgefüllt wird.
Die Magnetfeldstärke an der Resonanzstelle berechnet sich wie folgt:
\begin{align}
  \label{eq:2}
    h \nu &= E_{\text{HF}} = \mu_{\text{B}} \text{g}_{\text{J}} \cdot \text{B}_{\text{res}} \\
    \text{B}_{\text{Res}} &= \frac{4\pi m_0}{e_0 \text{g}_{\text{J}}}f \; .
\end{align}
\FloatBarrier
\begin{figure}
    \centering
    \includegraphics[width=0.8\textwidth]{Resonanz.PNG}
    \caption{Resonanzstellen des Alkali-Atoms in Abhängigkeit vom angelegten Magnetfeld.}
    \label{abb:Resonanz}
\end{figure}
\FloatBarrier

\subsection{Betrachtung für nicht verschwindende Kernspins}
Der nicht verschwindende Kernspin sorgt für weitere Aufspaltungen der Zeeman-Niveaus.
Wird dieses System mit zirkular polarisiertem Licht bestrahlt, so kommt
es lediglich zu Übergängen mit $\Delta \text{M}_{\text{F}} = +1$.
Dies wiederum hat zur Folge, dass lediglich der Zustand $\text{F}=2$ mit $M_F=+2$ besetzt sein wird, da von hier aus kein Übergang mehr induziert werden kann, da kein Niveau mit $\text{M}_{\text{F}}=+3$ existiert und dieser Zustand lediglich durch spontane Emission immer weiter besetzt werden kann.
Somit wird auch hier das Grundniveau "leer gepumpt".

\subsection{Quadratischer Zeeman-Effekt}
Bei stärkeren Magnetfeldern, müssen bei der Berechnung von $E_{\text{HF}}$ Terme höherer Ordnung berücksichtigt werden.
Die Eigenwertgleichung
\begin{align*}
    H \Psi = E \Psi
\end{align*}
wird unter Berücksichtigung der magnetischen Momente $\vec{\mu}_{\text{J}}$ und $\vec{\mu}_{\text{I}}$ und unter Abhängigkeit der Orientierungsquantenzahl $\text{M}_{\text{F}}$ lässt sich dann die Energie wie folgt beschreiben:
\begin{align}
  \label{eq_qZeeman}
    E_{\text{HF}} &= \text{g}_{\text{F}} \mu_{\text{B}} B+ \text{g}_{\text{F}}^2 \mu_{\text{B}}^2 B^2 \frac{1- \text{M}_{\text{F}}}{\Delta E_{\text{Hy}}}- \cdots  \; .
\end{align}
Hierbei ist $E_{\text{Hy}}$ ist die Energiedifferenz der Niveaus F und F$+1$ ist.

\subsection{Transiente Effekte}
Im Falle, dass sich das angelegte Magnetfeld periodisch ändert, müssen einige Betractungen angepasst werden.
In diesem Fall sollte die Frequenz, mit der sich das Magnetfeld ändert auf die Resonanzstelle eingestellt werden.
Die Frequenz ist dann gegeben durch:
\begin{align*}
    \omega_0 &= 2\pi f_0 \\
    &= \text{g}_{\text{F}} \frac{\mu_{\text{B}}}{h} \text{B}_0 \\
    &= \gamma \text{B}_0 \; ,
\end{align*}
wobei $\gamma$ das gyromagnetische Verhältnis ist.
Das Problem lässt sich als Summe von einem konstanten Magnetfeld und dem
Magnetfeld ${}^\omega\!/\!_\gamma$ auffasen, was durch das Lösen einer klassischen Differentialgleichung zeigen lässt.
Das effektive Magnetfeld lässt sich somit wie folgt berechnen:
\begin{align*}
    \text{B}_{\text{Eff}} = \text{B} + \frac{\omega}{\gamma} \; .
\end{align*}
Im Resonanzfall gilt |$\text{B}_{\text{Eff}} = \text{B}_{\text{res}}$|, woraus sich ergibt, dass es mit der Larmorfrequenz $f= \gamma_{\text{res}}$
rotiert.

\section{Auswertung}

\subsection{Berechnung der Horizontalkomponente des Erdmagnetfelds}
Zur Berechnung des B-Feldes müssen die gemessenen Umdrehungen in einen Strom
$I$ umgerechnet werden, wobei der Umrechnungsfaktor des Sweepfeldes 1 Umdrehung
= \SI{0,1}{\ampere} und der des Horizontalfeldes 1 Umdrehung = \SI{0,3}{\ampere}
beträgt.
Durch die Helmholtzgleichung \eqref{eq:1} kann aus dem Strom $I$ das angelegte
Magnetfeld berechnet werden. Die Messwerte und die daraus berechneten,
überlagerten Magnetfelder sind
in Tabelle \ref{tab:1} zu sehen.
\begin{equation}
  \label{eq:1}
  B = \mu_0 \cdot \frac{8IN}{\sqrt{125}R}
\end{equation}
(Magnetfeld $B$, magnetische Feldkonstante $\mu_0$, Strom $I$, Windungszahl $N$
und Radius $R$)

\begin{table}
  \centering
  \caption{Messwerte und berechnete Magnetfelder}
  \label{tab:1}
  \begin{tabular}{c|ccc}
    \toprule
    $\nu$/\si{\kilo\hertz} & \text{Sweep}/ \text{Umdrehungen} &
    \text{horizontales B-Feld} / \text{Umdrehungen} & $B$/\si{\micro\tesla} \\
    \midrule
    & \multicolumn{3}{l}{1. Isotop} \\
    \midrule
    98,8   &  6,04 &  7,2   & 36,45 \\
    200    &  4,02 &  6,61  & 29,52 \\
    298    &  6,09 &  9,57  & 42,01 \\
    402    &  3,81 &  8,55  & 59,82 \\
    498    &  2,02 &  7,92  & 72,70 \\
    602    &  1,08 &  8,21  & 85,45 \\
    702    &  1,17 &  9,46  & 101,77  \\
    800    &  1,51 &  7,01  & 114,34  \\
    900    &  2,92 &  8,38  & 128,12  \\
    1006   &  5,70 &  7,23  & 144,90  \\
    \midrule
    & \multicolumn{3}{l}{2. Isotop} \\
    \midrule
    98,8  &  7,2   &  0     &  43,45 \\
    200   &  6,61  &  0,02  &  45,15 \\
    298   &  9,57  &  0,02  &  63,01 \\
    402   &  8,55  &  0,14  &  88,43 \\
    498   &  7,92  &  0,23  &  108,31 \\
    602   &  8,21  &  0,30  &  128,47 \\
    702   &  9,46  &  0,36  &  151,80 \\
    800   &  7,01  &  0,49  &  171,22 \\
    900   &  8,38  &  0,55  &  195,27 \\
    1006  &  7,23  &  0,66  &  217,27 \\
    \bottomrule
  \end{tabular}
\end{table}

Zur Berechnung des horizontalen Erdmagnetfeldes wird eine lineare Regression der
Form
\begin{equation*}
  \nu = m \cdot B + b
\end{equation*}
durchgeführt. Die Steigungen und $y$-Achsenabschnitte sind in Tabelle \ref{tab:2}
aufgelistet. Daraus folgt für das horizontale Erdmagnetfeld
Erdmagnetfeld
\begin{align*}
  B_{\symup{Erde, horizontal, 1}} &= b_1 = \SI{10(4)}{\micro\tesla}\\
  B_{\symup{Erde, horizontal, 2}} &= b_2 = \SI{9(4)}{\micro\tesla}\\
\end{align*}

\begin{table}
  \centering
  \caption{Parameter der Ausgleichsrechnungen}
  \label{tab:2}
  \begin{tabular}{c c c}
    \toprule
    & $m$ / \si{\tesla\second} & $b$ / \si{\tesla} \\
    \midrule
    1. Isotop & \num{1,31(7)e-10} & \num{10(4)e-6} \\
    2. Isotop & \num{2,03(7)e-10} & \num{9(4)e-6} \\
    \bottomrule
  \end{tabular}
\end{table}

\begin{figure}
  \centering
  \includegraphics[scale=0.7]{Bfeld.pdf}
  \caption{Messwerte und lineare Regression.}
  \label{abb:1}
\end{figure}

\subsection{Berechnung der landéschen Faktoren und des Kernspins}
Die landéschen Faktoren werden ebenfalls durch die oben berechnete
lineare Regression bestimmt, indem die Regression mit Gleichung \eqref{eq:3}
verglichen wird:
\begin{align*}
  g_{F1} &= \frac{h}{\mu_B m_1} = \num{0,547(28)} \\
  g_{F2} &= \frac{h}{\mu_B m_2} = \num{0,352(12)} \\
\end{align*}

Zur Berechnung des Kernspins muss zunächst durch Gleichung \eqref{eq:La-Fa.gj}
der Faktor $g_J$ berechnet werden mit $J=S=0,5$, $L=0$ und $g_S = 2,0023$.
Daraus lässt sich dann der Kernspin $I$ mit der Gleichung \eqref{eq:La-Fa.gf}
berechnen:
\begin{align*}
  I_1 &= \num{1,33(10)} \\
  I_2 &= \num{2,35(10)} \\
\end{align*}
Aus dem Vergleich mit den Literaturwerten lässt sich somit eine Verbindung
zu den verwendeten Isotopen herstellen. $^{85}\text{Rb}$ (mit einem Kernspin
$I=5/2$) lässt sich der zweiten Resonanzstelle und $^{87}\text{Rb}$
($I=1/2$) der erste Resonanzstelle zuordnen.

\subsection{Signalbild}
In Abbildung \ref{abb:2} sind die zwei Resonanzstellen eines typischen Signalbildes
zu erkennen.

\begin{figure}
  \centering
  \includegraphics[scale=0.3]{Foto.png}
  \caption{Signalbild mit Resonanzstellen.}
  \label{abb:2}
\end{figure}

Aus dem Amplitudenverhältnis der beiden Resonanzstellen lässt sich auf das
Isotopenverhältnis schließen:
\begin{align*}
  A_1 &= 195\text{px} \\
  A_2 &= 387\text{px} \\
\end{align*}
Damit folgt, dass die Verteilung bei $\approx \SI{33,5}{\percent}$
($^{87}\text{Rb}$) und $\approx \SI{66,5}{\percent}$ ($^{85}\text{Rb}$) liegt.
In der Natur liegen die Isotope in dem
Verhältnis $\approx \SI{72}{\percent}$ ($^{87}\text{Rb}$) und
$\approx \SI{28}{\percent}$ ($^{85}\text{Rb}$) vor.

\subsection{Abschätzung des quadratischen Zeeman-Effekts}
Zur Abschätzung des quadratischen Zeeman-Effekts wird zunächst der lineare Teil
berechnet, um diesen mit dem quadratischen Teil zu vergleichen. Hierbei wird das
maximal gemessene Feld zur Berechnung genommen ( \SI{144,90}{\micro\tesla} und
\SI{217,27}{\tesla} für die jeweiligen Resonanzstellen). Mit $M_F=3$ für
$^{85}\text{Rb}$ und $M_F=2$ für $^{87}\text{Rb}$ und den
Hyperfeinstrukturaufspaltungen lässt sich mittels Gleichung \eqref{eq_qZeeman}
die linearen und quadratischen Teile berechnen:
\begin{align*}
  E_{1, linear} &= \SI{4,59(24)e-9}{\eV} \\
  E_{1, linear} &= \SI{4,42(15)e-9}{\eV} \\
  E_{1, quadratisch} &= \SI{-2,23(23)e-12}{\eV} \\
  E_{1, quadratisch} &= \SI{-7,8(5)e-12}{\eV} \\
\end{align*}
