\section{Theorie}
Der Germaniumdetektor wird zur $\gamma$-Spektroskopie verwendet, da dieser
gegenüber eines Szintillations-Detektors eine viel höhere Auflösung besitzt.
(Szintialltionsdetektor: $\approx \SI{70}{\kilo\eV} $,
Germaniumdetektor:$\approx \SI{895}{\kilo\eV} $ \cite{Q1})

\subsection{Wechselwirkung von Gamma-Strahlung mit Materie}
In Abbildung \ref{abb:1} sind die verschiedenen Wechselwirkungen mit Materie zu
erkennen. Es wurde dabei der Extinktionskoeffizient, ein Maß für die Schwächung
der elektromagnetischen Welle, gegen die Energie der $\gamma$-Quanten aufgetragen.
Zur Beschreibung der Wechselwirkungen, ist es notwendig auch den Wirkungsquerschnitt
zu betrachten. Dieser beschreibt die Wahrscheinlichkeit der Wechselwirkung mit
der Materie, diese hängt von der Energie der $\gamma$-Quanten und der
Eigenschaften des Materials ab. Die verschiedenen Wechselwirkungen werden im
Folgenden näher erklärt:

Bei dem Photoeffekt gibt das $\gamma$-Quant die komplette Energie
an ein Hüllenelektron (meist aus der K-Schale) eines Atoms ab. Übersteigt die
Energie des $\gamma$-Quants
die Bindungsenergie des Elektrons, so wird dieses aus dem Atom
herausgelöst. Die übrige Energie erhält das Elektron als kinetische Energie.
Da bei dem Herauslösen des Elektrons ein Loch entsteht, wird dieses durch ein
anderes Elektron einer höheren Schale aufgefüllt. Dabei wird Energie in Form
von $\gamma$-Strahlung frei, welche ein weiteres Elektron aus einer höheren
Schale herauslösen kann. Dieses zweite herausgelöste Elektron wird Auger-Elektron
genannt.
Der Wirkungsquerschnitt $\sigma_{\symup{Ph}}$ für den Photoeffekt ist stark
abhängig von der Energie  $E$ der $\gamma$-Quanten und der Kernladungszahl $Z$ des
Materials:
\begin{equation}
  \sigma_{\symup{Ph}} \propto Z^{\alpha}E^{\delta}
\end{equation}
wobei die Exponenten empirisch festgestellt wurden ($4<\alpha<5$ und
$\delta \approx -3,5$. Für Energien ab \SI{5}{\mega\eV} kann $\delta$ auf
einen Wert bis zu -1 steigen.

Eine weitere wichtige Wechselwirkung ist der Compton-Effekt.
Hierbei streut das $\gamma$-Quant elastisch an einem quasi freien Elektron und gibt
einen Teil seiner Energie an dieses in Form von kinetischer Energie ab.
Die Streuung bewirkt eine Energieabnahme und Richtungsänderung des
$\gamma$-Quants, wobei ein maximaler Energieübertrag bei einer Richtungsänderung
von $\Psi_{\gamma} = \pi$ stattfindet:
\begin{equation}
  E_{\gamma}' = E_{\gamma} \cdot ( 1+\epsilon(1-\cos{\Psi_{\gamma}}))^{-1} \;
  \symup{mit} \; \epsilon = \frac{E_{\gamma}}{m_0c²}
\end{equation}
Die Herleitung des Wirkungsquerschnitts $\sigma_{\symup{Co}}$ für den Compton-Effekt
ist sehr lang und kompliziert und wird hier nicht weiter erläutert.
Aus dieser Herleitung folgt:
\begin{equation}
  \label{eq:WQ}
  \sigma_{\symup{Co}}=\frac{3}{4}\sigma_{\symup{Th}}
  \left( \frac{1+\epsilon}{\epsilon^2} \left[ \frac{2+2\epsilon}{1+2\epsilon} -
  \frac{1}{\epsilon} \ln{(1+2\epsilon)} \right] +
  \frac{1}{2\epsilon}\ln{(1+2\epsilon)} -
  \frac{1+3\epsilon}{(1-2\epsilon)^2}\right)
\end{equation}
mit
\begin{equation}
  \sigma_{\symup{Th}} = \frac{8}{3}\pi \left(\frac{e_0}{4 \pi \epsilon_0 c^2 m_o} \right)^2
  := \frac{8}{3}\pi r_e^2
\end{equation}
wobei $e_o$ die Elementarladung, $c$ die Vakuumlichtgeschwindigkeit,
$\epsilon_0$ die Influenzkonstante, $m_0$ die Masse eines Elektrons und
$r_e$ den klassischen Elektronenradius darstellt.

Bei der Paarerzeugung wird ein Elektron und ein Positron erzeugt.
Hierbei muss ein $\gamma$-Quant aufgrund der Impulserhaltung an einem Potential
eines Atomkerns oder eines Elektrons gestreut werden, um ein Elektron und ein
Positron erzeugen zu können.
Bei einem Atom als Stoßpartner muss die Energie des $\gamma$-Quants
mindestens gleich der Ruheenergie von einem Elektron und einem Positron sein,
da hierbei die Energie in Masse umgewandelt wird. Der Impuls wird hierbei
an das Atom weitergegeben. Bei einem Elektron als Stoßpartner muss das
$\gamma$-Quant mindestens die vierfache Ruheenergie eines Elektrons besitzen,
da hier der Impuls nicht komplett an das Stoßelektron übergeben werden kann,
aufgrund der geringen Masse des Elektrons.
Der Wirkungsquerschnitt $\sigma_{\symup{Pa}}$ der Paarerzeugung kann mit viel
Aufwand hergeleitet werden. Hier sind zwei exemplarische Ergebnisse dargestellt,
wobei unterschieden werden muss für kernnahe und kernferne Paarbildung.
Für die kernnahe Paarbildung, also Bereiche von 10-25\,\si{\mega\eV}, gilt
die folgende Gleichung:
\begin{equation}
   \sigma_{\symup{Pa}}= \alpha r_e^2Z^2 \left( \frac{28}{9} \ln{(2 \epsilon)} -
   \frac{218}{27} \right)
\end{equation}
Für die kernferne Paarbildung (außerhalb der Elektronenhülle) lässt sich der
Wirkungsquerschnitt wie folgt berechnen:
\begin{equation}
  \sigma_{\symup{Pa}}= \alpha r_e^2Z^2 \left( \frac{28}{9} \ln{\left(\frac{183}{\sqrt[3]{Z}}\right)} -
  \frac{2}{27} \right)
\end{equation}
Diese Gleichung gilt für Energien die größer sind als \SI{500}{\mega\eV}.


\begin{figure}
  \centering
  \includegraphics[scale=0.8]{Wechselwirkungen.png}
  \caption{Effekte bei der Wechselwirkung von $\gamma$-Strahlung mit Materie. \cite{Q2}}
  \label{abb:1}
\end{figure}

\subsection{Aufbau und Wirkungsweise eines Germaniumdetektors}
Der Halbleiter-Detektor besteht im Wesentlichen aus einer Halbleiterdiode,
welche aus zwei Schichten besteht: Der n-dotierten Schicht als
Elektronen-Donator und einer p-dortierten Schicht als Elektronen-Akzeptor.
In dem Bereich zwischen den beiden Schichten entsteht eine Sperrschicht. Durch
Rekombination der Löcher aus der p-Schicht und der Elektronen aus der n-Schicht,
entsteht hier eine Zone ohne freie Ladungsträger.
Die benannten Löcher sind Elektronenlöcher, welche hier wie positiv geladene
Teilchen angesehen werden können. Diese entstehen durch p-Dotierung von Germanium
(4 Valenzelektronen) mit Bohr, welche nur 3 Valenzelektronen besitzt. Dadurch
entsteht durch die Einbindung von Bohr in das Gitter des Germaniums ein
Elektronenloch, bzw. eine nicht besetzte Elektronenstelle.
Die Sperrschicht ist wichtig für die Funktionsweise eines Germaniumdetektors,
da hier die Wechselwirkung zwischen den $\gamma$-Quanten und der Materie
stattfindet, bzw. nur hier die Wechselwirkung zu einem Zählereignis führt.
In der Sperrschicht werden Elektronen durch die vorher benannten Wechselwirkungen
mit sehr hohen kinetischen Energien freigesetzt. Diese widerum regen andere
Elektronen an, die zwar geringere kinetische Energien haben, aber genug um die
Lücke zwischen dem Leitungsband und dem Valenzband zu überspringen. So
entstehen in der Sperrschicht bzw. ladungsträgerarmen Zone Löcher und Elektronen,
die durch eine außen anliegende Spannung getrennt werden können, ohne wieder zu
rekombinieren. Durch diesen Vorgang entsteht ein Stromfluss, welcher zu einem
Zählereignis führt.
Um ein möglichst genaues Messergebnis zu erhalten, ist es sinnvoll die
ladungsträgerarme Zone zu verbreitern, denn die Breite beträgt ohne äußere Einflüsse
nur einige \si{\micro\meter}. Eine hohe außen angelegte Spannung (einige \si{\kilo\volt}), vergrößert
die ladungsträgerarme Zone auf einige \si{\centi\meter}. Dies ist nur begrenzt
möglich, da sich die Halbleiterdiode durch die hohe Spannung erhitzt und somit mehr
thermische Elektronen freigesetzt werden, die somit auch in das Leitungsband
übergehen können und somit für einen Strom sorgen, der die Messergebnisse
verfälscht. Deswegen wird während der Messung der Germaniumkristall mit
flüssigem Stickstoff auf $\SI{77}{\kelvin}$ gekühlt.
Außerdem ist die Dicke $d$ der Sperrschicht abhängig von der Dotierung
der Schichten, wie in Gleichung \eqref{eq:1} zu sehen.

\begin{equation}
  \label{eq:1}
  d = d_p + d_n \approx d_p \approx \sqrt{\frac{2 \epsilon \epsilon_0}
  {e_0}(U_D+U) \frac{1}{n_A}},
\end{equation}
wobei $d_p$ die Breite der Sperrschicht in der p-dotierten Schicht, $d_n$
die Breite der Sperrschicht in der n-dotierten Schicht, $U$ die angelegte
Sperrspannung und $U_D$ das Diffusionspotential beschreibt.
Um die Dicke also zu vergrößern ist es sinnvoll, die Akzeptorendichte $n_A$ so
gering wie möglich zu halten. Deswegen wird für einen Germaniumdetektor ein
sehr reiner Germanium-Kristall mit $n_A \approx 10^{10}\,
\frac{\symup{Atome}}{\symup{cm}} $ verwendet.

Der schematische Aufbau eines Germaniumdetektors ist in Abbildung \ref{abb:3}
zu sehen.

\begin{figure}
  \centering
  \includegraphics[scale=0.5]{AufbauDetektor.png}
  \caption{Schematischer Aufbau eines Germaniumdetektors. \cite{Q1}}
  \label{abb:3}
\end{figure}

Als p-dotierte Schicht und zusätzlich als Anschluss für die Hochspannung wird
eine dünne Goldfolie auf die Innenseite des Germaniumskristalls aufgetragen.
Als n-dotierte Schicht ist der Germaniumkristall auf der Außenseite von einer
Lithiumschicht umgeben. Der Detektor ist außerdem mit einer Aluminium
Schutzhaube umgeben üfr eine thermische Isolation. Zusätzlich wird die natürliche
Untergrundstrahlung durch eine Blei-Schicht abgeschirmt, um die Messergebnisse
nicht zu verfälschen.
Um gegebenenfalls strahlende Isotope aus dem Blei abzuschirmen, wird noch eine
weitere Schicht aus Kupfer unter der Blei-Schicht angebracht.

\subsection{Energiespektrum}
In Abbildung \ref{abb:4} ist das Energiespektrum eines monochromatischen
$\gamma$-Strahlers zu sehen.
\begin{itemize}
  \item Der Photopeak ensteht durch vollständige Abgabe der Enerige
  durch den Photoeffekt.
  \item Das Compton-Kontinuum wird durch die Comptonstreuung hervorgerufen.
  Hierbei wird nur ein Teil der Energie abgegeben, deswegen ist dies für die
  Energiebestimmung des $\gamma$-Quants nicht nützlich. Die Comptonkante entsteht durch
  den maximalen Energieübertrag bei einer Richtungsänderung des Photons von $\pi$.
  Nach der Compton-Kante ist dennoch eine Intensität zu erkennen. Diese kommt
  durch mehrfach compton-gestreute $\gamma$-Quants zustande.
  \item Der Rückstreupeak ensteht auch durch Comptonstreuung, aber
  nicht in dem Detektor, sondern in der Umgebung, beispielsweise in der
  Abschirmung.
\end{itemize}

\begin{figure}
  \centering
  \includegraphics[scale=0.5]{Spektrum.png}
  \caption{Spektrum eines monochromatischen $\gamma$-Strahlers. \cite{Q1}}
  \label{abb:4}
\end{figure}

\subsection{Energetisches Auflösungsvermögen und Effizienz}

Eine Kenngröße für das energetische Auflösungsvermögen ist die Halbwertsbreite
$\Delta E_{1/2}$ der Impulsverteilung bei monochromatischer $\gamma$-Strahlung,
denn Spektrallinien mit einem Energieunterschied von $\Delta E_{1/2}$
können noch gut voneinander unterschieden werden.
Die Halbwertsbreite wird durch die Anzahl der erzeugten
Elektronen-Loch-Paare durch ein $\gamma$-Quant festgelegt.
Außerdem werden bei der Erzeugung von einem Elektronen-Loch-Paar zusätzlich
Phononen erzeugt. Das kann aus der Differenz der Bildungsenergie des
Elektronen-Loch-Paars (hier etwa \SI{2,90}{\eV}) und der Energielücke zwischen
dem Leitungs- und Valenzband (hier etwa \SI{0,67}{\eV}) gefolgert werden, da
diese Energiedifferenz statistisch auf das Elektronen-Loch-Paar und die Phononen
verteilt wird.

Da die Ladungsträgerverteilung nicht komplett unkorreliert verläuft, sondern
die Fluktuation der Ladungsträgererzeugung durch die Fluktuation der
Phononenanregung kompensiert wird, wird der Fano-Faktor $F$ eingeführt ($F<1$).
Bei diesem Versuch liegt der Fano-Faktor bei etwa $F=0,1$
Damit folgt für die Standardabweichung $\sigma$
\begin{equation}
  \sigma = \sqrt{F\overline{n}} = \sqrt{F \frac{E_{\gamma}}{E_{\symup{El}}}}
\end{equation}
(Fano-Faktor $F$, Energie der $\gamma$-Quanten $E_{\gamma}$, Bildungsenergie der
Elektronen-Loch-Paare $ E_{\symup{El}}$).
Da die Anzahl der erzeugten Elektronen-Loch-Paare $n>>1$ ist, kann die
Poissonverteilung mit einer Gaußverteilung angenähert werden:
\begin{equation}
  \Delta E_{1/2} = \sqrt{2\ln{(2)}}\sigma_E =\sqrt{8\ln{(2)}}\frac{\sigma}{\overline{n}}
  E_{\gamma} = 2,35\sqrt{0,1E_{\gamma}E_{\symup{El}}}
\end{equation}
Die Halbwertsbreite des Germanium-Detektors bei einer typischen Energie von
\SI{500}{\kilo\eV} liegt bei $\Delta E_{1/2}(\SI{500}{\eV})\approx \SI{895}{\eV}$.

Durch einige Einflüsse wird das energetische Auflösungsvermögen vermindert:
\begin{itemize}
  \item Rauschen des Leckstroms, durch Eigenleitung und Restverunreinigung.
  Dies wird durch Abkühlen der Apperatur verringert.
  \item Rauschen des Verstärkers
  \item unvollständige Ladungssammlung durch Feldinhomogenitäten
\end{itemize}
Diese Störeinflüsse addieren sich unkorreliert zur theoretischen Halbwertsbreite
dazu.

Die Effizienz bezeichnet die Energieabhängigkeit der Nachweiswahrscheinlichkeit
bei der Detektion von $\gamma$-Quanten.
Die Effizienz Q eines Germaniumdetektors kann mit Hilfe einiger Parameter berechnet werden:
\begin{equation}
    \label{eq:effizienz}
    \text{Q} = \frac{4 \pi \text{Z}}{\Omega \text{A} \text{W}}
\end{equation}
Hierbei beschreibt A die Aktivität der Probe am Versuchstag, W die Emissionswahrscheinlichkeit, Z das gemessene Zählergebniss als Summe der Impulse in einem Peak und $\Omega$ dem Raumwinkel.
Das Raumwinkelelement $\frac{\Omega}{4\pi}$ berechnet sich wie folgt:
\begin{equation}
    \label{eq:Omega}
    \frac{\Omega}{4 \pi} = \frac{1}{2} \left(1- \frac{a}{\sqrt{a^2 + r^2}} \right)
\end{equation}

\subsection{Aufbau}

Der Aufbau des Versuchs ist in Abbildung \ref{abb:2} zu sehen. Folgende
Komponenten sind in dem Aufbau enthalten:
\begin{itemize}
  \item \textbf{Detektor}: Zur Detektion der $\gamma$-Strahlen mit Hilfe
  einer Halbleiterdiode.
  \item \textbf{Vorverstärker}: Der Vorverstärker integriert die vom Detektor
  kommende Ladungsmenge in einen Spannungsimpuls, da Spannung mit geringeren
  Verlusten weitergeleitet werden kann als Strom. Damit durch die Integration
  theoretisch keine unendlich hohe Spannung ensteht, muss der Kondensator am
  Vorverstärker nach jedem Ladungsimpuls entladen werden. Da durch eine
  Entladung über einen Widerstand ein Rauschen entsteht, wird der Kondensator
  durch eine optoelektrische Rückkopplung entladen, bei der die Gate-Drain-Schicht
  des Operationsverstärkers mit einer LED beleuchtet wird, wobei die
  Sperrschicht kurz leitend wird und die LAdung somit abfließen kann.
  \item \textbf{Hauptverstärker}: Der Hauptverstärker verstärkt die Spannung
  nach dem Vorverstärker. Wobei darauf geachtet werden muss, dass die Bandbreite
  des Operationsverstärkers weder zu klein noch zu groß ist, um alle Signale zu
  übertragen und das Rauschen nicht zu groß werden zu lassen.
  Zusätzlich differenziert und integriert der Hauptverstärker, als Hoch- und
  Tiefpassfilter.
  \item \textbf{Vielkanalanalysator}: Im Vielkanalanalysator sortiert
  die Spannungsimpulse nach ihrer Größe und zählt diese, indem ein Kondensator
  durch den zu messenden Spannungsimpuls geladen wird und die Zeit der
  Entladung erfasst wird. Der Kondensator ist zusätzlich an eine
  Gleichspannungsquelle angeschlossen. Da die Entladezeit ein Maß für die
  Höhe des Spannungsimpulses ist, wird die Zählerstelle des zur Entladezeit
  passenden Kanals um eins erhöht. Zur Messzeit wird das Gatter zu dem Kondensator
  geschlossen, damit kein zweiter Spannungsimpuls während der Messung
  den Kondensator auflädt und somit die Messung verfälscht.
  Die Messungen des Vielkanalanalysators werden auf dem angeschlossenen
  Computer gespeichert.
  \item \textbf{Temperaturregler}: Während der Messung wird die Aperatur
  durchgehgend durch flüssigen Stickstoff gekühlt, aufgrund der oben genannten
  Störstellen.
\end{itemize}

Um keine Verstärkung von Gleichspannungen und Offsetspannungen zu erhalten, wird
zwischen dem Vorverstärker und dem Hauptverstärker ein RC-Glied geschaltet. Ein
unerwünschter Nebeneffekt sind Unterschwingungen, welche zur Verschlechterung
des 1auflösungsvermögen führen. Um dies zu vermeiden wird ein Teil der
Gleichspannung über das RC-Glied geleitet ("Pole-Zero-Kompensation"). Das gleiche
Verfahren wird am Ausgang des Hauptverstärkers durchgeführt
("Base-Line-Restorer").

\begin{figure}
  \centering
  \includegraphics[scale=0.4]{Aufbau.png}
  \caption{Versuchsaufbau. \cite{Q1}}
  \label{abb:2}
\end{figure}



\section{Durchführung}
Zur Durchführung des Versuches wird zunächst die Gleichspannung langsam auf \SI{5}{\kilo\volt}
erhöht.
Die Probe und der Detektor befinden sich in einem mit Blei abgeschirmten Kasten,
um die Hintergrundstrahlung abzuschirmen. Außerdem ist die Bleischicht im Inneren
noch zusätzlich mit einer Kupferschicht umgeben, um gegebenenfalls strahlende
Blei-Isotope aus der Bleischicht abzuschirmen.
Nacheinander werden die jeweiligen Proben mit Hilfe eines Abstandhalters
vor dem Detektor befestigt. Die Messung wird für jede Probe je eine Stunde lang
durchgeführt.

\section{Auswertung}
Um die Funktionsweise des Germaniumdetektors zu untersuchen, werden nun im Folgenden die aufgenommenen Spektren ausgewertet.

\subsection{Kalibrierung und Effizienzbestimmung des Germaniumdetektors}

Zur Kalibrierung des Germaniumdetektors wird das Spektrum eines $^{152}$Eu-Strahlers untersucht. Hierbei werden im Spektrum die Peaks detektiert und dann den im Spektrum von $^{152}$Eu hervorstechenden Energiewerten zugeordnet. Das $^{152}$Eu-Spektrum mit den detektierten Peaks ist in Abbildung \ref{abb:Europiumspektrum} zu sehen. Die fertige Zuordnung von Binwerten zu den Energien ist in Tabelle \ref{tab:Kali} zu sehen.
\FloatBarrier
\begin{figure}
    \centering
    \includegraphics[scale=0.7]{Detektormesswerte.pdf}
    \caption{$^{152}$Eu-Spektrum mit detektierten Peaks.}
    \label{abb:Europiumspektrum}
\end{figure}
\FloatBarrier

\begin{table}
    \centering
    \caption{Werte zur Kalibrierung des Germaniumdetektors.}
    \label{tab:Kali}
    \begin{tabular}{ c c c c }
        \toprule
        {$\text{E}_{\gamma}$ / $si{\kilo \electronvolt}$} & { Wahrscheinlichkeit W} & {zugeordneter Bin-Index} & {Peakhöhe}     \\
        \midrule
        121,78 &    28,6 &  309,0 & 4608,0 \\
        244,7 &     7,6 &   614,0 & 692,0 \\
        295,94 &    0,4 &   711,0 & 35,0 \\
        344,3 &     26,5 &  861,0 & 1388,0 \\
        411,12 &    2,2 &   1027,0 & 94,0 \\
        443,96 &    3,1 &   1108,0 & 149,0 \\
        678,0 &     2,0 &   1689,0 & 29,0 \\
        688,67 &    0,9 &   1715,0 & 35,0 \\
        778,9 &     12,9 &  1939,0 & 184,0 \\
        867,37 &    4,2 &   2159,0 & 62,0 \\
        964,08 &    14,6 &  2398,0 & 134,0 \\
        1005,3 &    0.6 &   2500,0 & 15,0 \\
        1085,9 &    10,2 &  2702,0 & 85,0 \\
        1112,1 &    13,6 &  2766,0 & 106,0 \\
        1299,1 &    1,6 &   3228,0 & 10,0 \\
        1408,0 &    21,0 &  3502,0 & 116,0 \\
        1457,6 &    0,5 &   3633,0 &   6,0 \\
        \bottomrule
    \end{tabular}
\end{table}

\noindent Mittels der Zuordnung der Bin-Indizes zu den Energien kann nun eine lineare Ausgleichsrechunung der Form
\begin{align*}
    E = m \cdot x + n
\end{align*}
durchgeführt werden.
Für die Werte des linearen Fits ergibt sich dann:

\begin{align*}
    m &= \SI{0,4018 \pm 0,0007}{\kilo \electronvolt} \\
    n &= \SI{0,2(16)}{\kilo \electronvolt}
\end{align*}

\noindent Die zugeordneten Bin-Indizes und Energien sind gemeinsam mit dem linearen Fit in Abbildung \ref{abb:linfit} zu sehen.
\FloatBarrier
\begin{figure}
    \centering
    \includegraphics[scale=0.7]{Kalibrierung.pdf}
    \caption{Energiekalibrierung des Germaniumdetektors.}
    \label{abb:linfit}
\end{figure}
\FloatBarrier

\noindent Um die Effizienz des Germaniumdetektors gemäß der Formel \ref{eq:effizienz} zu bestimmen, müssen zunächst die Parameter A, Z, W und $\Omega$ ermittelt werden.
Die Aktivität A der Probe am Versuchstag, dem 07.11.2018 lässt sich mit den Angaben zur Halbwertszeit von $^{152}$Eu ($t_{1/2}=\SI{4943(5)}{\days}$) und der Aktivität am 01.10.2000: $\text{A}_0= \SI{4130(60)}{\becquerel}$ aus der Versuchsanleitung \cite{Q1} einfach berechnen:
\begin{align*}
    \text{A} &= \text{A}_0 \exp \left(-\frac{\ln(2) \Delta t}{t_{1/2}}\right)
    &= \SI{1634(24)}{\becquerel}
\end{align*}
Das Raumwinkelelement $\frac{4 \pi\Omega}{4 \pi}$ berechnet sich nach Gleichung \ref{eq:Omega} mit dem Parameter $a$ für den Abstand zwischen Quelle und Absorptionspunkt ($a=\SI{8,81}{\centi\meter}$) und $r$ für die halbe Breite des Germaniumdetektors ($r= \SI{2,25}{\centi \meter}$). Die Wahl für $r$ ergibt sich aus der Angabe aus der Anleitung zum Versuch \cite{Q1}, dass der wahrscheinlichste Absorptionspunkt im Germanium bei $\SI{1,5}{\centi \meter}$ und der Höhe des Abstandshalters zwischen Probe und Detektor $\SI{7,31}{\centi \meter}$.
Mit Hilfer dieser Parameter berechnet sich das Raumwinkelelement zu:
\begin{align*}
    \frac{\Omega}{4 \pi} = 0,016 \; .
\end{align*}

\noindent Die Peakinhalte zu den angegebenen Energien wird bestimmt, indem Gaußfits über jeden Peak gelegt werden. Die Gaußfits besitzen folgende Form:
\begin{equation*}
    f(x) = h \cdot \exp\left(\frac{(x-\mu)^2}{2\sigma^2}\right) +a \;
\end{equation*}
Hierbei beschreibt $h$ die Höhe des Peaks, $\mu$ den Mittelwert, mit dem der zuvor zugeordnete Bin-Index korrigiert wird, $\sigma$ die Standardabweichung und $a$ die Untergrundstrahlung.
Die Peakinhalte $Z_i$ werden dann über Integration über die Gaußkurven berechnet:
\begin{equation*}
    Z_i = \sqrt{2\pi} h_i \sigma_i
\end{equation*}
\FloatBarrier
Die sich ergebenden Parameter und Ergebnisse für die Peakinhalte $Z_i$, sowie die berechneten Effizienzen ($Q(E_i)$) sind in den Tabellen \ref{tab:effizienz1} und \ref{tab:effizienz2} zu sehen.
Anschließend wird ein nicht linearer Fit der Werte für $E_i$ und $Q(E_i)$ vorgenommen, der folgende Form besitzt:
\begin{equation*}
    Q(E) = b \cdot (E-c)^d + e
\end{equation*}
Hierbei ist zu beachten, dass lediglich Energien berücksichtigt werden, die größer als $\SI{150}{\kilo \electronvolt}$.
Für die Fitparameter ergeben sich folgende Werte:
\begin{align*}
    b &= 0,014 \pm 0,11 \\
    c &= 246,79 \pm 123,14 \\
    d &= 6,54 \pm 3,57 \\
    e &= 0,84 \pm 1,06  \\
\end{align*}
Die Wertepaare, als auch die gefittete Potenzfunktion sind in Abbildung \ref{abb:effizienz} zu sehen.
\begin{table}
    \centering
    \caption{Werte zur Effizienbestimmung aus den Gaußfits}
    \label{tab:effizienz}
    \begin{tabular}{ c c c c c }
        \toprule
        {$\sigma_i$} & {$\mu_i$} & {$h_i$} & {$a$}  \\
        \midrule

        1.130 \pm 0.005   &   308.800 \pm 0.005     & 4579 \pm 18       & 90\pm4            & (12970 \pm 80           \\
        1.293 \pm 0.014   &   613.806 \pm 0.013     & 652 \pm 6         & 38.5\pm1.3        & 2113 \pm 29             \\
        (0\pm9) $10^3$     &   (1\pm4)$10^3$        & (0\pm5)$10^6$     & 26.6\pm0.9        & (0.0 \pm 3.3)$10^6$       \\
        1.547 \pm 0.014   &   860.911 \pm 0.013     & 1349 \pm 10       & 20.1\pm2.6        & 5230 \pm 60             \\
        1.80 \pm 0.08     &   1026.69 \pm 0.07      & 80.6 \pm 2.9      & 17.5\pm0.8        & 364\pm21                \\
        1.58 \pm 0.06     &   1108.16 \pm 0.06      & 125 \pm 4         & 15.8\pm1.0        & 496\pm24                \\
        1.3 \pm 0.40       &   1689.2 \pm 0.4        & 12.0 \pm 3.5      & 14.5\pm0.8        & 39\pm17                 \\
        1.01 \pm 0.27     &   1715.07 \pm 0.26      & 19 \pm 4          & 14.6\pm0.9        & 49\pm17                 \\
        2.58 \pm 0.06     &   1939.19 \pm 0.05      & 166.9 \pm 3.1     & 12.0\pm1.1        & 1080\pm32               \\
        2.44 \pm 0.17     &   2158.90 \pm 0.16      & 47.7 \pm 2.7      & 13.0\pm1.0        & 292\pm26                \\
        3.24 \pm 0.13     &   2398.51 \pm 0.11      & 123 \pm 4         & 6.5\pm1.7         & 1000\pm 50              \\
        1.01 \pm 0.34     &   2500.23 \pm 0.33      & 9.1 \pm 2.6       & 6.1\pm0.5         & 23\pm10                 \\
        3.18 \pm 0.14     &   2700.87 \pm 0.12      & 75.9 \pm 2.6      & 9.8\pm1.1         & 604\pm34                \\
        3.35 \pm 0.13     &   2766.14 \pm 0.11      & 98.5 \pm 3.0      & 5.7\pm1.4         & 830\pm 40               \\
        0.83 \pm 0.22     &   3228.31 \pm 0.20      & 9.1 \pm 2.0       & 2.40\pm0.35       & 19\pm6                  \\
        3.63 \pm 0.15     &   3501.10 \pm 0.12      & 103.2 \pm 3.2     & 2.4\pm1.6         & 940\pm 50               \\
        6.2\pm1.6       &   3630.4\pm1.0            & 3.4 \pm 0.6       & 0.0\pm0.6         & 53\pm17                 \\
        \bottomrule
    \end{tabular}
\end{table}

\FloatBarrier
\begin{table}
    \centering
    \caption{Werte zur Effizienbestimmung aus den Gaußfits}
    \label{tab:effizienz}
    \begin{tabular}{ c c c }
        \toprule
        {$Z_i$} & {$E_i$ / $\si{\kilo\electronvolt}$} & {$Q(E_i)$ / $\si{\per\becquerel}$} \\
        \midrule
        12970\pm 80           &      124.2375\pm0.0021  & 17.86\pm0.28        \\
        2113\pm29             &     246.795\pm0.005     & 10.94\pm0.22        \\
        (0.0\pm3.3)$10^6$     &       (0.3\pm1.7)$10^3$ & (0.0\pm3.3)$10^5$     \\
        5230\pm60             &     346.086\pm0.005     & 7.78\pm0.14         \\
        364\pm21              &     412.700\pm0.030     & 6.5\pm0.4           \\
        496\pm24              &     445.436\pm0.022     & 6.30\pm0.32         \\
        39\pm17               &     678.91\pm0.17       & 0.76\pm0.34         \\
        49\pm17               &     689.30\pm0.11       & 2.1\pm0.7           \\
        1080\pm32             &     779.360\pm0.022     & 3.29\pm0.11         \\
        292\pm26              &     867.64\pm0.06       & 2.73\pm0.25         \\
        1000\pm 50            &     963.92\pm0.05       & 2.70\pm0.14         \\
        23\pm10               &     1004.80\pm0.13      & 1.5\pm0.7           \\
        604\pm34              &     1085.42\pm0.05      & 2.33\pm0.13         \\
        830\pm 40             &     1111.64\pm0.04      & 2.39\pm0.12         \\
        19\pm6                &     1297.35\pm0.08      & 0.47\pm0.16         \\
        940\pm 50             &     1406.96\pm0.05      & 1.76\pm0.09         \\
        53\pm17               &     1458.9\pm0.4        & 4.2\pm1.3           \\
        \bottomrule
    \end{tabular}
\end{table}

\begin{figure}
    \centering
    \includegraphics[scale=0.7]{effizienz.pdf}
    \caption{Effizienz des Germaniumdetektors gegen die Energie aufgetragen.}
    \label{abb:effizienz}
\end{figure}

\subsection{Bestimmung einiger Detektoreigenschaften an Hand eines $^{137}$Cs-Strahlers}

Um gewisse Detektoreigenschaften zu untersuchen wird das Spektrum eines monochromatischen Gammastrahlers, das von $^{137}$Cs aufgenommen. Das erhaltene Spektrum ist zusammen mit den detektierten Peaks in Abbildung \ref{abb:Cs_fit} zu sehen. Die für die Auswertung wichtigen Peaks sind der Photopeak, der Rückstreupeak, sowie die Compton-Kante. Die erhaltenen Werte für die Bin-Indizes, sowie die zugehörigen Energiewerte sind in Tabelle \ref{tab:Cs_peaks} dargestellt.
\FloatBarrier
\begin{table}
    \centering
    \caption{Lage und Energie der charakteristischen Peaks des $^{137}$Cs-Strahlers}
    \label{tab:Cs_peaks}
    \begin{tabular}{ c c c }
        \toprule
        & {Bin-Index} & {Energie $E_i$}      \\
        \midrule
        {Rückstreupeak} & 486  & 195,44      \\
        {Compton-Kante} & 1164 & 467,87      \\
        {Photopeak}     & 1648 & 662,35      \\
        \bottomrule
    \end{tabular}
\end{table}
\FloatBarrier
Bei der Analyse der Lage der Peaks ist zu erkennen, dass die Energie des Photopeaks mit $\SI{662,35}{\kilo \electronvolt}$ sehr nah am Theoriewert ($\SI{661,59}{\kilo \electronvolt}$) liegt \ref{Q3}.
Der Theoriewert für Rückstreupeak und Compton-Kante kann wie folgt aus der Energie des Photopeaks berechnet werden:
\begin{align*}
    E_{\text{comp,theo}} &= E_\text{photo} \frac{2\epsilon}{1+2\epsilon} = \SI{477,98}{\kilo\electronvolt} \\ \Delta E_\text{comp} &= \SI{2,11}{\percent} \\
    E_{\text{rück,theo}} &= E_\text{photo} \frac{1}{1+2\epsilon} = \SI{184,38}{\kilo\electronvolt} \\
    \Delta E_\text{rück} &= \SI{6,00}{\percent}
\end{align*}
In dieser Berechnung beschreibt $\epsilon$ das Verhältnis zwischen der Energie des Photopeaks $E_\text{photo}$ zur Ruheenergie eines Elektrons: $\epsilon = \frac{E_\text{photo}}{m_e c^2}$.

\noindent Im Anschluss daran wird übverprüft, ob der Photopeak Gauß-förmig ist.
Dies geschieht, indem die Flanken des Peaks mittels linearer Ausgleichsrechnung approximiert werden. Somit kann leicht die Halberts- und Zehntelwertsbreite ermittelt werden. Stehen die beiden Breiten in einem bestimmten Verhältnis zueinander, so kann davon ausgegangen werden, dass der Peak Gauß-förmig ist.
Die lineare Ausgleichsrechung hat die Form $F(x) = m x + n$.
Für die Parameter links (l) und rechts (r) am Peak ergeben sich folgende Werte:
\begin{align*}
    m_l &= 0,0028 \pm 0,0003 & m_r &= -0,00213 \pm 0,00018 \\
    n_l &= 1641,9 \pm 0,4    & n_r &=  1652,94 \pm 0,25
\end{align*}
Die Berechnung der Halbwerts- und Zehntelwertsbreite erfolgt mittels folgender Rechnung:
\begin{align*}
    \Delta_x = m_r x h + n_r - (m_l x h +n_l) \; \text{mit} \; h \in \{0.1, 0.5\} \;
\end{align*}
und ergibt folgende Werte:
\begin{align*}
    \Delta_{0,5} &= \SI{2,45(23)}{\kilo \electronvolt}\\
    \Delta_{0,1} &= \SI{4,16(18)}{\kilo \electronvolt} \; .
\end{align*}
Für eine Gaußverteilung gilt:
\begin{align*}
    \Delta_{0,1} &= 1,823 \cdot \Delta_{0,5} \\
                 &= \SI{4.3(4)}{\kilo \electronvolt} \; .
\end{align*}
Der mit Hilfe der Halbwertsbreite berechnete Wert für die Zehntelwertsbreite weicht dabei lediglich um $\SI{4}{\percent}$ von dem experimentell ermittelten ab, weshalb davon auszugehen ist, dass eine Gaußverteilung vorliegt.
\FloatBarrier
\begin{figure}
    \centering
    \includegraphics[scale=0.7]{Cs_fit.pdf}
    \caption{Comptonkontinuum und Photopeak des $^{137}$Cs-Strahlers.}
    \label{Comptonkontinuum und Photopeak des $^{137}$Cs-Strahlers.$}
\end{figure}
\FloatBarrier

\noindent Im letzten Teil wird der Inhalt des Photopeaks und des Comptonkontinuums mit der Absorptionswahrscheinlichkeit für Photo- und Comptoneffekt verglichen.
Die Extinktionskoeffizienten $\mu$ ergeben sich aus der Versuchsanleitung \cite{Q1}, woraus anschließend die Absorptionswahrscheinlichkeiten berechnet werden können ($p = 1-\exp(-\mu l)$). Hierbei beschreibt $l= \SI{3,9}{\centi \meter}$ die Länge des Detektors.
\begin{align*}
    \mu_{\text{comp}} &= \SI{0,38}{\centi \meter} \\
    p_{\text{comp}} &= \SI{77,28}{\percent}\\
    \mu_{\text{photo}} &= \SI{0,002}{\centi \meter} \\
    p_{\text{photo}} &= \SI{0,78}{\percent}
\end{align*}
Der Inhalt des Photopeaks wird auch in diesem Teil des Versuchs ermittelt, indem er mit einem Gaußfit gefittet wird. Das Comptonkontinuum hingegen wird mit einer Funktion gefittet, die proportional zum zugehörigen Wirkungsquerschnitt \ref{eq:WQ} ist. Die Proportionalitätskonstante wird durch das Verhältnis von der Zählrate an der Compton-Kante zum Wert der Funktion definiert.
Die Funktion, mit der das Comptonkontinuum gefittet wurde, wird anschließend mittels der Funktion "quad" aus scipy.integrate integriert, sodass sich für den Inhalt des Comptonkontinuums und des Photopeaks folgende Werte ergeben:
\begin{align*}
    Z_{\text{comp}}  &= 21079,33 \\
    Z_{\text{photo}} &= 5487 \pm 29
\end{align*}
Es ist zu sehen, dass der Inhalt des Photopeaks zwar kleiner als der des  Comptonkontinuums ist, jedoch nicht derart kleiner, wie es die deutlich unterschiedlich großen Absorptionswahrscheinlichkeiten hätten erwarten lassen.
Somit kann davon ausgegangen werden, dass die Elektronen nicht nur einmalig Energie beim Comptoneffekt abgeben, sondern mehrfach, bevor sie dann ihre komplette Energie im durch den Photoeffekt verlieren.
Der Extinktionskoeffizient \ref{abb:1} ist höher für den Photoeffekt, was diesen wahrscheinlicher macht.

\subsection{Aktivität einer $^{133}$Ba-Quelle}

\noindent In diesem Teil des Versuchs wird das Spektrum einer $^{133}$Ba-Strahlers untersucht. Dafür werden die Peaks zunächst analog zum ersten Teil des Versuchs detektiert, anschließend Gauß-gefittet, die Peakinhalte ermittelt, indem über die Gaußfits integriert wird. Die zugeordneten Bin-indizes, sowie gefitteten Energien sind in Tabelle \ref{tab:BaTab} dargestellt.
\FloatBarrier
\begin{table}
    \centering
    \caption{Zugeordnete Peaks und Bins einer $^{133}$Ba-Quelle. }
    \label{tab:BaTab}
    \begin{tabular}{cccc}
        \toprule
        {E / keV} & { W } & zugeordneter Bin-Index & gefittete Energie \\
        \midrule
        53.16   &   2.2   &   139.0  &    55.86\pm0.07 \\
        79.62   &   2.6   &   192.0  &    71.3\pm2.4 \\
        81.0    &   34.1  &   208.0  &    83.575\pm0.008 \\
        160.61  &   0.6   &   405.0  &    163.00\pm0.09 \\
        302.85  &   18.3  &   758.0  &    304.7824\pm0.0026 \\
        356.02  &   62.1  &   890.0  &    357.8077\pm0.0030 \\
        383.85  &   8.9   &   960.0  &    385.616\pm0.010 \\
        \bottomrule
    \end{tabular}
\end{table}
\FloatBarrier

\noindent Mit Hilfe der Formel für die Ermittlung der Effizienz \ref{eq:effizienz}, kann so ein Wert für die Aktivität der Bariumquelle ermittelt werden. Hierbei ist darauf zu achten, da nur Werte mit einer Energie $E_i$ größer als $\SI{150}{\kilo \electronvolt}$, da nur für diese Werte die Effizienz bestimmt werden kann.
Die Werte, die sich für die Bin-Indizes, die Standardabweichung, die Höhe der Peaks, die Peakinhalte, die Energie, sowie die Aktivität der Quelle ergeben, sind in den Tabellen \ref{tab:BaAkt} \ref{tab:BaAkt2} dargestellt.
Die Detektormesswerte, sowie die ermittelten Peaks der Bariumquelle sind in Abbildung \ref{abb:BaPlot} dargestellt.
\FloatBarrier
\begin{table}
    \centering
    \caption{Ergebnisse zur bestimmung der Aktivität der $^{133}$Ba-Quelle aus den Gaußfits der Peaks.}
    \label{tab_BaAkt}
    \begin{tabular}{ c c c }
        \toprule
        {$\sigma_{\text{ba}}$} & {$h_i$} &  {$a_i$}                     \\
        \midrule
        0.85 \pm 0.19     & 64 \pm 12              &     71.6 \pm 1.5   \\
        -13 \pm 9         & -470 \pm 210           &     470 \pm 170    \\
        1.081 \pm 0.020   & 3490\pm 50             &     84 \pm 8       \\
        0.79 \pm 0.23     & 84 \pm 21              &     89.3 \pm 2.6   \\
        1.384 \pm 0.007   & 930 \pm 4              &     12.9 \pm 0.7   \\
        1.513 \pm 0.008   & 2403 \pm 10            &     8.0 \pm 1.9    \\
        1.657 \pm 0.027   & 286 \pm 4              &     2.9 \pm 0.8    \\
        \bottomrule
    \end{tabular}
\end{table}
\FloatBarrier
\begin{table}
    \centering
    \caption{Ergebnisse zur bestimmung der Aktivität der $^{133}$Ba-Quelle aus den Gaußfits der Peaks.}
    \label{tab:BaAkt2}
    \begin{tabular}{ c c c }
        \toprule
        {Bin-Index} & {Peakinhalt $Z_i$} & {Aktivität $A_i$} \\
        \midrule
        138.62 \pm 0.17   & 130 \pm 40       & - \\
        170 \pm 8         & 11000 \pm 1000   & - \\
        207.604 \pm 0.017 & 9480 \pm 200     & - \\
        405.26 \pm 0.23   & 190 \pm 70       & - \\
        758.119 \pm 0.006 & 3227 \pm 18         & 1858\pm11 \\
        890.082 \pm 0.007 & 9120 \pm 50         & 1633\pm10 \\
        959.288 \pm 0.023 & 1187 \pm 22         & 1525\pm28 \\
        \bottomrule
    \end{tabular}
\end{table}
\FloatBarrier
Die erhaltenen Werte für die Aktivität werden anschließend gemittelt, wodurch sich folgender Wert für die Aktivität der $^{133}$Ba-Quelle ergibt:
\begin{align*}
    A_\text{Ba} = \SI{1672\pm18}{\becquerel}
\end{align*}
\FloatBarrier
\begin{figure}
    \centering
    \includegraphics[scale=0.7]{Ba_plot_peaks.pdf}
    \caption{Spektrum mit detektierten Peaks einer $^{133}$Ba-Quelle.}
    \label{abb:BaPlot}
\end{figure}
\FloatBarrier
\subsection{Aktive Nuklide eines unbekannten Strahlers}
Im finalen Teil des Versuchs wird eine unbekannte Substanz im Germaniumdetektor analysiert. Das Spektrum, sowie die detektierten Peaks sind in Abbildung \ref{abb:unbekannt} zu sehen. Die Position, sowie die Höhe und zugehörige Energie sind in Tabelle \ref{tab:unbekannt} zu sehen.
\FloatBarrier
\begin{figure}
    \centering
    \includegraphics[scale=0.7]{unbekannterStrahler.pdf}
    \caption{Emissionsspektrum mit detektierten Peaks des unbekannten Strahlers.}
    \label{abb:unbekannt}
\end{figure}
\FloatBarrier
\begin{table}
    \centering
    \caption{Detektierte Peaks und zugeordnete Energie des unbekannten Strahlers.}
    \label{tab:unbekannt}
    \begin{tabular}{ c c c }
    \toprule
    {Bin-Index} & {counts } & {Energie $E_i$ / $\si{\kilo\electronvolt}$}\\
    \midrule
    2918 & 153.0 & 1172.66       \\
    3313 & 113.0 & 1331.38       \\
    \bottomrule
    \end{tabular}
\end{table}
\FloatBarrier

\noindent Die Energien der ermittelten Peaks werden anschließend mit Literaturwerten für verschiedene Gammastrahler verglichen. Wichtigstes Merkmal des unbekannten Gammastrahlers sind in diesem Spektrum die einzigen zwei deutlich hervorstechenden Peaks bei $\SI{1172,66}{\kilo \electronvolt}$ und $\SI{1331,38}{\kilo \electronvolt}$. Diese beiden Peaks passen sehr gut zu den beiden Peaks des $^{60}$Co, die laut Literaturangaben \cite{Q4} $\SI{1,173}{\mega \electronvolt}$ und $\SI{1,332}{\mega \electronvolt}$ betragen sollen.

\section{Diskussion}
Im ersten Teil des Versuchs konnte der lineare Zusammenhang zwischen Channel-Eingang und der Energie gut sichtbar gemacht werden.
Die große Abweichung im Fitparameter $n$ ist eventuell dadurch erklärbar, dass sich die niedrigen Peaks mit wenigen Counts nicht gut über Gaußfits approximieren lassen. Außerdem werden für die Erstellung des Fits die unkorrigierte Channelnummern verwendet.
Die sehr großen Fehler im nicht linearen Fit zur Bestimmung der Eiffizienz des Detektors lassen sich ebenso durch die schlecht zu fittenden
niedrigen Peaks erklären. Die großen Abweichungen bei der Bestimmung der Effizienz des Detektors führen dann weiterführend zu großen Abweichungen bei der Bestimmung der Aktivität von $^{133}$Ba.
Die hohe Präzision des Germaniumdetektors ist an der sehr geringen Abweichung von $\SI{0,1}{\percent}$ des detektierten Wertes für die Energie des Photopeaks vom Literaturwert \cite{Q3}. Ebenso spricht die Abweichung der Compton-Kante mit lediglich zwei Prozent vom Theoriewert für die hohe Präzision des Detektors. Die etwas höhere Abweichung des Theoriewertes für den Rückstreupeak ist dadurch zu erklären, dass der theoretische Wert nur als eine Näherung zu betrachten ist und das Ablesen des Peaks eine gewisse Ungenauigeit birgt.
Die Emissionslinien des unbekannten Strahlers konnten deutlich dargestellt und mit den Theoriewerten verglichen werden.

\printbibliography
