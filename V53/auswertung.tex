\section{Auswertung}
\subsection{Vermessung der Moden}
Zur Untersuchung dreier Moden werden jeweils die Reflektorspannungen an den rechten ($U_2$)
und linken Nullstellen ($U_1$), sowie im Maximum ($U_0$) ermittelt.
Die Ergebnisse sind gemeinsam mit den Amplituden der drei Moden in Tabelle
\ref{tab:Moden} zu sehen. In Abbildung \ref{fig:Moden} sind die Messwerte graphisch
zusammen mit den angedeuteten Modenkurven abgebildet.
\FloatBarrier
\begin{table}
    \centering
    \begin{tabular}{c c c c c c}
        \toprule
        Mode & Reflektorspannung & Reflektorspannung & Reflektorspannung & Amplitude & Frequenz \\
        {}     & {$U_0$ \,/\,$\si{V}$} & {$U_1$ \,/\,$\si{V}$} & {$U_2$ \,/\,$\si{V}$ }& {$A$ \,/\,$\si{V}$ }& {$f_0$\, / \, $\si{MHz}$} \\
        \midrule
        Mode 1  &  215  &  200  &  230  &  0,124  &  9000     \\
        Mode 2  &  135  &  120  &  150  &  0,134  &  9005     \\
        Mode 3  &   80  &   70  &   90  &  0,108  &  9010     \\
        \bottomrule
    \end{tabular}}
    \caption{Messwerte für die Reflektorspannungen, bei denen jeweils die rechte und linke Nullstelle, sowie das Maximum
    auf dem Referenzpunkt im Oszillographen lagen und die zugehörigen Amplituden und Frequenzen.}
    \label{tab:Moden}
\end{table}
\FloatBarrier

\noindent Die Messwerte können durch eine Funktion folgender Form gefittet werden, um die
Modenkurven graphisch darzustellen:
\begin{align*}
    A(U) = x \cdot U^2 + y \cdot U + z \; .
\end{align*}

\noindent Die Fitparameter, die sich für die drei Moden ergeben, sind in Tabelle
\ref{tab:Modenfit} zu sehen. Da die Fitfunktion eine Parabel darstellt und ledigleich
drei Messwerte genommen werden, können keine Fehler angegeben werden, da sich eine
Parabel stets fehlerfrei durch drei Punkte legen lässt.
\FloatBarrier
\begin{table}
    \centering
    \begin{tabular}{c c c c}
        \toprule
            {}    &  {Paramter x}  &  {Parameter y } &  {Paramter  z }\\
        \midrule
        Mode 1 &  -0,00055  &  0,23698  &  -25,35111          \\
        Mode 2 &  -0,00060  &  0.16080  &  -10,72000          \\
        Mode 3 &  -0.00108  &  0.17280  &  -6.804000          \\
        \bottomrule
    \end{tabular}
    \caption{Fitparamter zur Erstellung des Plots \ref{fig:Moden}.}
    \label{tab:Modenfit}
\end{table}
\FloatBarrier
\begin{figure}
  \centering
  \includegraphics[width = 12 cm]{moden.pdf}
  \caption{Graphen der drei Moden.}
  \label{fig:Moden}
\end{figure}
Die Reflektorspannungen und Frequenzen, bei denen sich im Oszilloskop ein Einsattelung
im Maximun, sowie auf der linken oder rechten Flanke der Mode ergaben, sind in Tabelle
\ref{tab:elektronische_Abstimmung} zu sehen.
\FloatBarrier
\begin{table}
    \centering
    \begin{tabular}{c c c c}
        \toprule
        {}    & {Einsattelung auf dem Maximun & Einsattelung re. } & {Einsattelung li.} \\
        \midrule
        Reflektorspannung / $\si{\V}$ & 215 & 205 & 225                           \\
        Frequenz / $\si{\MHz}$ & 9000 & 8983 & 9021                               \\
        \bottomrule
    \end{tabular}
    \caption{Messwerte zur elektronischen Abstimmung des Klystrons.}
    \label{tab:elektronische_Abstimmung}
\end{table}
\FloatBarrier
Die Bandbreite $\Delta f_{E}$ wird mit folgender Formel berechnet:
\begin{align*}
    \Delta f_E = f^{\prime\prime} - f^{\prime}
\end{align*}
\FloatBarrier
und beträgt $\SI{38}{MHz}$.
Daraus lässt sich nun die Abstimmempfindlichkeit $E$ mittels folgender Formel berechnen:
\begin{align*}
    E &= \frac{\Delta f_{E}}{V^{\prime \prime}-V^{\prime}}
      &= \SI{1,9}{\MHz \per \V} \;.
\end{align*}

\subsection{Messung von Frequenz, Wellenlänge und Dämpfung}
Zur Bestimmung der Wellenlänge im Hohlleiter wurde die Lage zweier aufeinander
folgende Minima, sowie die eingestellte Frequenz aufgenommen. Die Werte finden
sich in Tabelle \ref{tab:Frequenz}. Der doppelte Abstand der Minima entspricht dabei
der Hohlleiterwellenlänge ($\lambda_{g}$) und ergibt sich mit den aufegnommenen Messwerten zu:
\begin{align*}
    \lambda_{g} = \SI{58}{\mm}
\end{align*}
\begin{table}
    \centering
    \begin{tabular}{c c c}
        \toprule
        {Frequenz / $\si{\MHz}$ } & {1. Minimum / $\si{\mm}$ } & { 2. Minimum / $\si{\mm}$} \\
        \midrule
        9010   &  84  &  55  \\
        \bottomrule
    \end{tabular}
    \caption{Messwerte zweier aufeinander folgender Minima und zugehöriger Frequenz.}
    \label{tab:Frequenz}
\end{table}
Mit Hilfe der Gleichung \ref{eq:Frequenz} und der Angabe für die Länge der Längsseite
des Hohlleiters ($a= \SI{22,860{46}}{\mm}$) aus der Versuchsanleitung \cite{Q1} lässt
sich die Frequenz im Hohlleiter berechnen:
\begin{align*}
    f = \SI{8355(10)}{\MHz} \; .
\end{align*}
\\
Die aufegnommenen Werte für die Dämpfung, die zugehörige Mikrometereinstellung,
sowie die aus der Eichkurve abgelesene Dämpfung sind in Tabelle \ref{tab:Daempfung}
zu sehen.
\FloatBarrier
\begin{table}
    \centering
    \begin{tabular}{c c c }
        \toprule
        {SWR-Meter Ausschlag / $\si{\dB}$ } & { Mikrometereinstellung / $\si{\mm}}$} & {Dämpfung aus Eichkurve / $\si{\dB}$ } \\
        \midrule
         0       &       3,38        &       22      \\
         2       &       3,60        &       24      \\
         4       &       3,81        &       26      \\
         6       &       3,98        &       29      \\
         8       &       4,07        &       30      \\
        10       &       4,26        &       33      \\
        \bottomrule
    \end{tabular}
    \caption{Gemessene Werte für die Dämpfung, Mikrometereinstellung, sowie abgelesene, zugehörige Werte aus der Eichkurve.}
    \label{tab:Daempfung}
\end{table}

\noinden Die graphische Darstellung der gemessenen Dämpfungswerte, sowie die Eichkurve ist in
Abbildung \ref{fig:Daempfung} zu sehen.
\FloatBarrier
\begin{figure}
    \includegraphics{Daempfung.pdf}
    \caption{Eichkurve und gemessene Werte für die Dämpfung.}
    \label{fig:Daempfung}
\end{figure}
\FloatBarrier

\subsection{Bestimmung des Stehwellenverhältnisses}
In Tabelle \ref{tab:SWR_Methode} sind die am SWR-Meter gemessenen Ausschläge für
das Stehwellenverhältnis in Abhängigkeit der Sondentiefe am Gleitschraubentransformator
zu sehen.
\FloatBarrier
\begin{table}
    \centering
    \begin{tabular}{c c}
        \toprule
        {SWR-Meter Ausschlag / $\si{\dB}$} & Sondentiefe / $\si{\mm}$ \\
        \midrule
        1,02    &   3       \\
        1,06    &   5       \\
        1,55    &   7       \\
        4,0     &   1,55    \\
        \bottomrule
    \end{tabular}
    \caption{Messwerte für die SWR-Methode.}
    \label{tab:SWR_Methode}
\end{table}
\FloatBarrier
Tabelle \ref{tab:3dB_Methode} zeigt die Messwerte, die zur Bestimmung des Stehwellenverhältnisses
mittels der 3dB Methode aufgenommen wurden. Aus der Lage der beiden Minima lässt sich zunächst
die Wellenlänge $\lambda_{g}$ berechnen. Mit Hilfe von Formel \ref{eq:3dB} lässt sich dann
das Stehwellenverhältnis berechnen.
\FloatBarrier
\begin{table}
    \centering
    \begin{tabular}{c c c c c c}
        \toprule
        {$d_1$\,/\,mm} & {$d_2$\,/\,mm} & {1. Minimum\,/\,mm} & {2. Minimum\,/\,mm} & {$\lambda_g$ / } & {genähertes Stehwellenverhältnis $S$}\\
        \midrule
        95,5    &   71,0    &   96,2    &   71,7    &   49    &     0,64
        \bottomrule
    \end{tabular}
    \caption{Messwerte zur Bestimmung des Stehwellenverältnisses mittels 3dB Methode.}
    \label{tab:3dB_Methode}
\end{table}
\FloatBarrier
Bei der letzten Methode zur Bestimmund des Stehwellenverhältnisses wurden die Dämpfungseinstellungen
$A_{1}= \SI{20}{\dB}$ und $A_{2}=\SI{43}{\dB}$ gemessen.
Aus den beiden Parametern ergibt sich nach Gleichung \ref{eq:Abschwaechermethode}
für das Stehwellenverhältnis ein Wert von:
\begin{align*}
    S_{\text{Abschw.}} = \SI{11,5}{} \; .
\end{align*}

\section{Diskussion}
Die im ersten Versuchsteil aufgenommenen Werte für die Reflektorspannung und die Amplituden
zeigen den zu erwartenden Verlauf. Hier sei darauf hingewiesen, dass die Parabeln
lediglich durch drei Messpunkte gefittet wurden, weshalb keine Fehler für die Fitparameter auftauchen.
Entsprechend der Theorie können Moden mit höheren Frequenzen bei niedrigeren Reflektorspannung festgestellt werden.
Die zunächst mit steigender Reflektorspannung größer werdende Amplitude entspricht der Theorie,
nicht so jedoch die dritte Amplitude, die bei höherer Reflektorspannung wieder kleiner wird.

\nodindent Bei der Frequenzbestimmung mittels der Wellenlänge im Hohlleiter ($\lambda_{g}$), fällt
eine Abweichung von $\SI{7,27}{\percent}$ auf. Diese kann durch das, am Oszilloskop schwierige
Auffinden der beiden Minima erklärt werden. Dennoch kann der experimentell ermittelte Wert als verhältnismäßig genau
eingestuft werden.

\noindent Bei der Betrachtung der mit dem SWR-Meter gemessenen Werte für die Dämpfung
zeigt sich in beiden Kurven ein sehr ähnlicher Verlauf, der sich lediglich in der Größe
der Funktionswerte unterscheidet. Somit kann auch hier davon ausgegangen werden, dass die
Messwerte plausibel sind.

\noindent
Zum Vergleich der drei Methoden zur Bestimmung des Stehwellenverhältnisses, werden die
drei Ergebnisse bei einer Sondentiefe von $\SI{9}{\mm}$ betrachtet:
\begin{align*}
    S_{\text{SWR-Meter}} &= 4,00
    S_{\text{3dB}} &= 0,64
    S_{\text{Abschw.}} &=11,5
\end{align*}
Es fällt auf, dass sich der Wert für die 3dB-Methode extrem stark von den anderen beiden
Werten unterscheidet, weshalb hier davon ausgegangen werden muss, dass ein Fehler bei der Bestimmung der
Minima entstanden sein muss.
Eine Begründung für den großen Wert, der bei der Abschwächermethode entstand, könnte sein,
dass kein Präzisionsabschwächer verwendet wurde. So kam es bei der Lokalisation der 3dB-Punkte zu
abrupten Vollausschlägen am SWR-Meter.
